\documentclass{beamer}

% http://deic.uab.es/~iblanes/beamer_gallery/index_by_theme.html
\mode<presentation>
{
  \usetheme{Madrid}       % or try default, Darmstadt, Warsaw, ...
  \usecolortheme{default} % or try albatross, beaver, crane, ...
  \usefonttheme{serif}    % or try default, structurebold, ...
  \setbeamertemplate{navigation symbols}{}
  \setbeamertemplate{caption}[numbered]
} 

\usepackage[english]{babel}
\usepackage[utf8x]{inputenc}
\usepackage{chemfig}
\usepackage[version=3]{mhchem}

\usepackage{hyperref}
\usepackage{natbib}
\bibliographystyle{agsm}

\usepackage{animate}
\usepackage{xmpmulti}


\title[Trend Segmentation]{Uses of High Dimensional Trend Segmentation}
\institute[LSE Department of Statistics]{}

\date[8 May 2017]{
  \hspace{1cm}\\
  PhD Presentation Event\\
  \hspace{1cm}\\
  \href{https://github.com/Shakeel95/PhD-Presentation-1}{GitHub}}

\begin{document}

\begin{frame}
  \titlepage
\end{frame}


\begin{frame}{Roadmap}
  \tableofcontents
\end{frame}


%%%%%%%%%%%%%%%%%%%%%%%%%%%
%% SECTION 1: Motivation %%
%%%%%%%%%%%%%%%%%%%%%%%%%%%

\section{Motivation}


%%% Table of contents %%%

\begin{frame}{Roadmap}
\tableofcontents[currentsection]
\end{frame}


\begin{frame}{Model Setup}

Univariate: $X_t = f(t) + \varepsilon_t$ with $\varepsilon_t \sim \left ( 0, \sigma^2 \right )$ and weakly dependent and $f(t)$ piecewise linear on $[1,T]$ having kinks at $\tau_1,..., \tau_N$: 

\begin{equation*}
    f(t) = 
    \left\{\begin{matrix}
            \theta_{1,\tau_1} + t \cdot \theta_{2,\tau_1} & t \in [1, \tau_1) \\ 
            \theta_{1,\tau_2} + t \cdot \theta_{2,\tau_2} & t \in [\tau_2, \tau_3)\\ 
            \vdots & \\ 
            \theta_{1,\tau_N} + t \cdot \theta_{2,\tau_N} & t \in [\tau_N,T]
    \end{matrix}\right.
\end{equation*}

Multivariate: $\boldsymbol{X_t} = \boldsymbol{F}(t) + \boldsymbol{\varepsilon}_t$ with $\boldsymbol{\varepsilon}_t \sim \left ( \boldsymbol{0}, \Sigma_{n \times n} \right )$

$\boldsymbol{X_t} = \left ( X_{1,t},...,X_{n,t}\right )'$ where each $X_j$ is as above and $\boldsymbol{F}(t) = \left ( f_1(t), ..., f_n(t)\right )'$ and the functions $\left \{ f_i\right \}_{i=1}^n$ share common kinks $\tau_1,..., \tau_N$.

\end{frame}


\begin{frame}

\begin{figure}[h]
    \centering
    \includegraphics[width=0.5\textwidth]{../plots/TGUW_gif/TGUW_1}
    \label{model 3 cov}
\end{figure}

\end{frame}

\begin{frame}
\begin{figure}[h]
	\centering
	\animategraphics[loop,controls,width=0.55\linewidth]{5}{../plots/TGUW_gif/TGUW_}{1}{96}
\end{figure}
\end{frame}

\end{document}
\end{frame}

\end{document}